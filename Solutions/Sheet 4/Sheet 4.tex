\documentclass[12pt]{scrartcl}
\usepackage[utf8]{inputenc}
\usepackage[german]{babel}
\usepackage[fleqn]{amsmath}
\usepackage{amsfonts}
\usepackage{amssymb}
\usepackage{color}
\usepackage{ebproof}
\usepackage{pdflscape}

\newcommand{\lbr}{[\![}
\newcommand{\rbr}{]\!]}

\usepackage[left=2cm,right=2cm,top=2cm,bottom=2cm]{geometry}
\begin{document}

\begin{center}
	\huge\textbf{Exersice Sheet 4}\\[0.5cm]
	
	\Large
	$\mathrel{\vcenter{\hbox{\rule{2cm}{0.5pt}}}}$ \textbf{Sample 				Solution} $\mathrel{\vcenter{\hbox{\rule{2cm}{0.5pt}}}}$\\[1cm]
\end{center}
	\large

	\section*{Task 1: Chain Complete Partial Orders}
		\subsection*{(a): \underline{TRUE}}
		\begin{enumerate}
			\item[] Let $d\sqsubseteq_{1}d'$. Then $\left\{d,\;d'\right\}$ is a chain.\\
			Thus $f\left(d'\right)=f\underbrace{\left(\sqcup_{1}\left\{d,\;d'\right\}\right)}_{=d'}\overset{f\; \text{continous}}{=}\sqcup_{2}\left\{f\left(d\right),\;f\left(d'\right)\right\}\overset{\text{def. L.U.B}}{\sqsupset_{2}}f\left(d\right)$.\\
			Therefore $f\left(d\right)\sqsubset_{2}f\left(d'\right)$ holds.\\
			Alternatively: True by Definition 7.15.
		\end{enumerate}
		
		\subsection*{(b): \underline{FALSE}}
		\begin{enumerate}
			\item[] Let $S=\left\{x\in\mathbb{Q}\; |\; x\leq\sqrt{2}\right\}$.\\
			Then $S$ is a chain, but $\sqcup S=\sqrt{2}\not\in \mathbb{Q}$.
		\end{enumerate}
		
		\subsection*{(c): \underline{FALSE}}
		\begin{enumerate}
			\item[] Let $\left(D_{1},\; \sqsubseteq\right)=\left(\mathbb{N}\cup\left\{\infty\right\},\;\leq\right)$ and\\
			$f:\;D_{1}\rightarrow D_{1},\; x\mapsto\left\{
			\begin{aligned}
				0, &\qquad x<\infty\\
				\infty &\qquad x=\infty			
			\end{aligned}\right.$.\\
			$f$ is monotonic, because $x\leq y\Rightarrow f\left(x\right)=\left\{\begin{aligned}
			0 &\quad \leq f\left(y\right) &\qquad \text{if}\; x<\infty\\
			\infty &\quad \leq f\left(y\right) &\qquad \text{if}\; x=y=\infty
			\end{aligned}\right.$.\\
			However, $\underbrace{f\left(\sqcup\mathbb{N}\right)}_{=\infty}\nleq\underbrace{\sqcup f\left(\mathbb{N}\right)}_{=0}$.
		\end{enumerate}
		
		\subsection*{(d): \underline{TRUE}}
		\begin{enumerate}
			\item[] Since $f\left(p\right)\sqsubseteq p$, it suffices to prove $p\sqsubseteq f\left(p\right)$.\\
			First note that $f\left(p\right)$ implies $f\left(f\left(p\right)\right)\sqsubseteq f\left(p\right)$ \null\hfill {\color{red}$\left(f\left(p\right)=p'\right)$}\\
			Since $p$ is the least element with $f\left(p\right)\sqsubseteq p$, we have $p\sqsubseteq p'=f\left(p\right)$.\\
			Thus $p=f\left(p\right)$ holds.
		\end{enumerate}		
	\section*{Task 2: repeat-until Loops}
	\subsection*{(a)}
	
	\begin{enumerate}
	\item[] 
		\begin{equation*}		
		\begin{split}
			  & \mathfrak{C}\lbr \text{repeat}\;c\;\text{until}\;b\rbr\\
			= & \mathfrak{C}\lbr c;\;\text{if}\; b\; \text{then}\;\text{skip}\; \text{else}\; \text{repeat}\; c\;\text{until}\; b\rbr\\
			= & \mathfrak{C}\lbr \text{if}\; b\; \text{then}\;\text{skip}\; \text{else}\; \text{repeat}\; c\;\text{until}\; b\rbr \circ \mathfrak{C}\lbr c\rbr\\
			= & \text{cond}\left( \mathfrak{B}\lbr b\rbr ,\; \mathfrak{C} \lbr \text{skip}\rbr ,\; \mathfrak{C}\lbr \text{repeat}\;c\; \text{until}\; b\rbr \right) \circ \mathfrak{C}\lbr c\rbr\\
			= & \text{cond}\left( \mathfrak{B}\lbr b\rbr ,\; id_\Sigma ,\; \mathfrak{C}\lbr \text{repeat}\;c\; \text{until}\; b\rbr \right) \circ \mathfrak{C}\lbr c\rbr\\
		\end{split}
		\end{equation*}
		
		Then $\mathfrak{C}\lbr \text{repeat}\;c\; \text{until}\; b\rbr$ is the least fixed point of F given by:
		\begin{equation*}
		F\left( f \right) = \text{cond}\left( \mathfrak{B}\lbr b\rbr ,\; id_\Sigma ,\; \mathfrak{C}\lbr \text{repeat}\;c\; \text{until}\; b\rbr \right) \circ \mathfrak{C}\lbr c\rbr
		\end{equation*}	
	\end{enumerate}
	
	\subsection*{(b)}
	
	\begin{enumerate}
	\item[] 
		\begin{equation*}
		\begin{split}
			\hat{F}\left(f\right) & =\text{cond}\left( \mathfrak{B}\lbr \text{false}\rbr ,\; id_\Sigma ,\; f \right) \circ \mathfrak{C}\lbr \text{skip}\rbr\\
			& = f \circ \mathfrak{C}\lbr \text{skip}\\
			& = f \circ id_\Sigma\\
			& = f
		\end{split}
		\end{equation*}
		$\hat{F}$ is the identity transformer.
	\end{enumerate}		
	
	\subsection*{(c)}
	
	\begin{enumerate}
	\item[] Since $\hat{F}\left(f\right)=f$ for all $f:\Sigma \dashrightarrow \Sigma$ and $f_{\emptyset}\sqsubseteq f$ for all $f:\Sigma\dashrightarrow \Sigma$, we have $\text{fix}\left(\hat{F}\right)=f_{\emptyset}$
	
	\end{enumerate}
	
	\section*{Task 3: Closed Sets}
	
	\subsection*{(a)}
	
	\begin{enumerate}
	\item[] Apply Tarski-Knaster Theorem:
	\[\text{fix}\left(f\right)=\sqcup\left\{f^{n}\left(\sqcup\emptyset\right)|n\geq 0\right\}\]
	Since $\emptyset$ is a chain and $\emptyset\subseteq C$, we have $\sqcup\emptyset\in C$.\\
	By definition, $f\left(\sqcup\emptyset\right)\in C$.\\
	By complete induction, $\forall n.f^{n}\left(\sqcup\emptyset\right)\in C$.\\
	Hence, $G=\left\{f^{n}\left(\sqcup\emptyset\right)|n\geq 0\right\}\subseteq C$ is a chain.\\
	By definition, $\text{fix}\left(f\right)=\sqcup G\in C$.
	\end{enumerate}
	
	\subsection*{(b)}
	
	\begin{enumerate}
	\item[] Let $C:=\left\{y\in D|y\sqsubseteq x\right\}$. C is closed.\\
	If $y\in D$ then $f\left(y\right)\sqsubseteq f\left(x\right)\sqsubseteq x$.\\
	Thus $f\left(y\right)\in C$. By (a), $\text{fix}\left(f\right)\in C$.\\
	By definition of $C$, $\text{fix}\left(f\right)\sqsubseteq x$.
	\end{enumerate}
	
\end{document}