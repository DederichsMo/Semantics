\documentclass[12pt]{scrartcl}
\usepackage[utf8]{inputenc}
\usepackage[german]{babel}
\usepackage[T1]{fontenc}
\usepackage{amsmath}
\usepackage{amsfonts}
\usepackage{amssymb}
\usepackage{amsthm}
\usepackage{color}
\usepackage{ebproof}
\usepackage{pdflscape}

\newcommand{\eRelation}[2]{$\dfrac{\text{#1}}{\text{#2}}$}
\newcommand{\eState}[2]{$\left\langle \text{#1},\; \text{#2}\right\rangle$}
\newcommand{\eRule}[3]{\eState{#1}{#2} $\rightarrow$ #3}

\newcommand{\sgabb}[2]{\sigma_{{#1},{#2}}}
\newcommand{\mytextcircled}[1]{\text{\textcircled{#1}}}

\newcommand{\Mod}[0]{\text{\textbf{mod}}}
\newcommand{\dep}[0]{\text{\textbf{dep}}}
\newcommand{\FV}[0]{\text{\textbf{FV}}}

\usepackage[left=2cm,right=2cm,top=2cm,bottom=2cm]{geometry}
\begin{document}

	\begin{center}
		\huge\textbf{Exersice Sheet 2}\\[0.5cm]
	
		\Large
		$\mathrel{\vcenter{\hbox{\rule{2cm}{0.5pt}}}}$ \textbf{Sample 				Solution} $\mathrel{\vcenter{\hbox{\rule{2cm}{0.5pt}}}}$\\[1cm]
	\end{center}
\large

	\section*{Task 1: Operational Semantics \& Derivation Trees}

	\indent\indent To shorten the derivation tree we first introduce the  following two\\\indent abbreviations.\\
	\indent\indent $c_{1}=\text{\textbf{while}}\; \left(x \leq y\right)\; \text{\textbf{do}}\; c_{2}\; \text{\textbf{end}}$\\
	\indent\indent $c_{2}= y\;:=\;y\;-\;x;\; x\;:=\;x\;-\;4$\\

	\indent Furthermore we introduce the notation $\sigma_{ij}$ which defines\\\indent $\sigma \left(x\right)=i$ and $\sigma \left(y\right)=j$.\\
	\normalsize
	\begin{center}		
		\begin{prooftree}
    \infer0{\langle 23, \sigma\rangle \rightarrow 23}
    \infer[left label = \small{(asgn)}]1{\langle x:=23, \sigma\rangle \rightarrow \sigma[x \mapsto 23]}
    \infer0{\langle 42, \sigma[x \mapsto 23]\rangle \rightarrow 42}
    \infer[left label = \small{(asgn)}]1{\langle y:=42, \sigma[x \mapsto 23]\rangle \rightarrow \sgabb{23}{42}}
    \hypo{\mytextcircled{a} \langle c_1, \sgabb{23}{42}\rangle \rightarrow \sgabb{15}{0}}
    \infer[left label = \small{(seq)}]2{\langle y:=42; c_1, \sigma[x \mapsto 23]\rangle \rightarrow \sgabb{15}{0}}
    \infer[left label = \small{(seq)}]2{\langle c,\sigma \rangle \rightarrow \sgabb{15}{0}}
		\end{prooftree}\vspace{1cm}
		
		\textcircled{a}
\hspace*{1cm}\begin{prooftree}
    \infer0{\langle x,\sgabb{23}{42}\rangle  \rightarrow 23}
    \infer0{\langle y,\sgabb{23}{42}\rangle  \rightarrow 42}
    \infer2{\langle x \leq y, \sgabb{23}{42}\rangle \rightarrow \text{\textbf{true}}}
    \hypo{\mytextcircled{b} \langle c_2, \sgabb{23}{42}\rangle \rightarrow \sgabb{19}{19}}
    \hypo{\mytextcircled{c} \langle c_1, \sgabb{19}{19}\rangle \rightarrow \sgabb{15}{0}}
    \infer[left label = \small{(wh-t)}]3{\langle c_1, \sgabb{23}{42}\rangle \rightarrow \sgabb{15}{0}}
\end{prooftree}\vspace{1cm}

\textcircled{b}
\hspace*{1cm}\begin{prooftree}
    \infer0{\langle y, \sgabb{23}{42}\rangle \rightarrow 42}
    \infer0{\langle x, \sgabb{23}{42}\rangle \rightarrow 23}
    \infer2{\langle y-x, \sgabb{23}{42}\rangle \rightarrow 19}
    \infer[left label = \small{(asgn)}]1{\langle y:=y-x, \sgabb{23}{42}\rangle \rightarrow \sgabb{23}{19}}
    \infer0{\langle x, \sgabb{23}{19}\rangle \rightarrow 23}
    \infer0{\langle 4, \sgabb{23}{19}\rangle \rightarrow 4}
    \infer2{\langle x-4, \sgabb{23}{19}\rangle \rightarrow 19}
    \infer[left label = \small{(asgn)}]1{\langle x:=x-4, \sgabb{23}{19}\rangle \rightarrow \sgabb{19}{19}}
    \infer[left label = \small{(seq)}]2{\langle c_2, \sgabb{23}{42}\rangle \rightarrow \sgabb{19}{19}}
\end{prooftree}\vspace{1cm}

\textcircled{c}
\hspace*{1cm}\begin{prooftree}
    \infer0{\langle x,\sgabb{19}{19}\rangle  \rightarrow 19}
    \infer0{\langle y,\sgabb{19}{19}\rangle  \rightarrow 19}
    \infer2{\langle x \leq y, \sgabb{19}{19}\rangle \rightarrow \text{\textbf{true}}}
    \hypo{\mytextcircled{d} \langle c_2, \sgabb{19}{19}\rangle \rightarrow \sgabb{15}{0}}
    \hypo{\mytextcircled{e} \langle c_1, \sgabb{15}{0}\rangle \rightarrow \sgabb{15}{0}}
    \infer[left label = \small{(wh-t)}]3{\langle c_1, \sgabb{19}{19}\rangle \rightarrow \sgabb{15}{0}}
\end{prooftree}\vspace{1cm}


\textcircled{d}
\hspace*{1cm}\begin{prooftree}
    \infer0{\langle y, \sgabb{19}{19}\rangle \rightarrow 19}
    \infer0{\langle x, \sgabb{19}{19}\rangle \rightarrow 19}
    \infer2{\langle y-x, \sgabb{19}{19}\rangle \rightarrow 0}
    \infer[left label = \small{(asgn)}]1{\langle y:=y-x, \sgabb{19}{19}\rangle \rightarrow \sgabb{19}{0}}
    \infer0{\langle x, \sgabb{19}{0}\rangle \rightarrow 19}
    \infer0{\langle 4, \sgabb{19}{0}\rangle \rightarrow 4}
    \infer2{\langle x-4, \sgabb{19}{0}\rangle \rightarrow 15}
    \infer[left label = \small{(asgn)}]1{\langle x:=x-4, \sgabb{19}{0}\rangle \rightarrow \sgabb{15}{0}}
    \infer[left label = \small{(seq)}]2{\langle c_2, \sgabb{19}{19}\rangle \rightarrow \sgabb{15}{0}}
\end{prooftree}\vspace{1cm}

\textcircled{e}
\hspace*{1cm}\begin{prooftree}
    \infer0{\langle x,\sgabb{15}{0}\rangle \rightarrow 15}
    \infer0{\langle y,\sgabb{15}{0}\rangle \rightarrow 0}
    \infer2{\langle x \leq y, \sgabb{15}{0}\rangle \rightarrow \text{\textbf{false}}}
    \infer[left label = \small{(wh-f)}]1{\langle c_1, \sgabb{15}{0}\rangle \rightarrow \sgabb{15}{0}}
\end{prooftree}
	\end{center}
\large
	
	\section*{Task 2: Operational Semantics of other Statements}
		\indent\indent For $c\in Cmd,\; \sigma , \sigma ^{'}, \sigma ^{''} \in \Sigma$ and $b\in BExp$ the \textcolor{red}{repeat until relation}\\\indent $\left< \text{\textbf{repeat}}\; c\; \text{\textbf{until}}\; b,\; \sigma \right> \rightarrow \sigma ^{''}$ is defined by:\\
		
		\begin{center}
			\eRelation
				{\eRule{c}{$\sigma$}{$\sigma ^{''}$ \quad \eRule{b}{$\sigma ^{''}$}{\textbf{true}}}}
				{\eRule{\textbf{repeat} c \textbf{until} b}{$\sigma$}{$\sigma ^{''}$}}
			(\textbf{repeat}-\textbf{true})\\[0.75cm]
			\eRelation
				{\eRule{c}{$\sigma$}{$\sigma ^{'}$} \quad \eRule{b}{$\sigma ^{'}$}{\textbf{false}} \quad \eRule{\textbf{repeat} c \textbf{until} b}{$\sigma ^{'}$}{$\sigma ^{''}$}}
				{\eRule{\textbf{repeat} c \textbf{until} b}{$\sigma$}{$\sigma ^{''}$}}
			(\textbf{repeat}-\textbf{false})\\
		\end{center}
	\section*{Task 3: Termination}
		\indent\indent Prove that \eRule{\textbf{while} b \textbf{do} c \textbf{end}}{$\sigma$}{$\sigma ^{'}$} implies that \eRule{b}{$\sigma ^{'}$}{\textbf{false}}.\\
		\indent This will be proven by induction on the height h of derivation trees.\\
		\subsection*{\underline{Induction Base: (h=1)}}
		\indent\indent If the derivation tree has height 1 only one derivation is possible, namely\\
		\begin{center}
			\eRelation
				{\eRule{b}{$\sigma$}{false}}
				{\eRule{\textbf{while} b \textbf{do} c \textbf{end}}{$\sigma$}{$\sigma ^{'}$}}(\textbf{while}-false)
		\end{center}
		\indent\indent Since this rule is unambiguous the induction base holds trivially.\\\newpage
		\subsection*{\underline{Induction Hypothesis:}}
		\begin{center}
			\eRule{\textbf{while} b \textbf{do} c \textbf{end}}{$\sigma$}{$\sigma ^{'}$} implies \eRule{b}{$\sigma ^{'}$}{\textbf{false}}
		\end{center}
		\indent\indent holds for all derivations of an arbitrary, but fixed height h and for all states \indent$\sigma , \; \sigma ^{'}$.\\
		\subsection*{\underline{Induction Step: ({\boldmath$h\mapsto h+1$})}}
		\indent\indent For all derivations of height $h+1$ ($h\geq 1$), we have\\
		\begin{center}		
		\begin{prooftree}
			\hypo{\left\langle b,\;\sigma \right\rangle\rightarrow \text{\textbf{true}}}
			\hypo{\left\langle c,\;\sigma\right\rangle\rightarrow\sigma^{'}}
			\hypo{\cdots \text{(derivation tree of height h)}}
			\infer1{\left\langle\text{\textbf{while}}\; b\;\text{\textbf{do}}\; c\;\text{\textbf{end}},\;\sigma^{'}\right\rangle\rightarrow\sigma^{''}}
			\infer3{\left\langle\text{\textbf{while}}\; b\;\text{\textbf{do}}\; c\;\text{\textbf{end}},\;\sigma\right\rangle\rightarrow\sigma^{''}}
		\end{prooftree}\\
		\end{center}
		\indent\indent By Induction Hypothesis \eRule{\textbf{while} b \textbf{do} c \textbf{end}}{$\sigma ^{'}$}{$\sigma ^{''}$} implies\\\indent \eRule{b}{$\sigma ^{'}$}{\textbf{false}}.\\
		\indent Due to the propagating characteristics of the derivation trees we also know \indent that \eRule{\textbf{while} b \textbf{do} c \textbf{end}}{$\sigma$}{$\sigma ^{''}$} implies \eRule{b}{$\sigma ^{''}$}{$\text{\textbf{false}}$}. 
		\begin{flushright}
			$\qed$
		\end{flushright}
	\section*{Task 4: Variables that do not matter}
	
	\subsection*{(a)}
	
	\indent\indent $\text{\textbf{mod}}: \qquad \text{\textbf{Cmd}}\rightarrow 2^{\text{\textbf{Var}}}$,\\
	\indent $\text{\textbf{skip}}\mapsto\emptyset$\\
	\indent $x:=a\mapsto \left\lbrace x \right\rbrace$\\
	\indent $c_1;c_2\mapsto \text{\textbf{mod}}\left( c_1\right) \cup\text{\textbf{mod}}\left( c_2\right)$\\
	\indent $\text{\textbf{repeat}}\; c\; \text{\textbf{until}}\; b \mapsto \text{\textbf{mod}}\left( c\right)$\\
	
	\subsection*{(b)}
	
	\indent\indent $\text{\textbf{dep}}: \qquad \text{\textbf{Cmd}}\rightarrow 2^{\text{\textbf{Var}}}$,\\
	\indent $\text{\textbf{skip}}\mapsto\emptyset$\\
	\indent $x:=a\mapsto \text{\textbf{FV}}\left( a\right)$\\
	\indent $c_1;c_2\mapsto \text{\textbf{dep}}\left( c_1\right) \cup\text{\textbf{dep}}\left( c_2\right)$\\
	\indent $\text{\textbf{repeat}}\; c\; \text{\textbf{until}}\; b \mapsto \text{\textbf{dep}}\left( c\right) \cup \text{\textbf{FV}}\left( b\right)$

	\subsection*{(c)}
	
	\indent\indent Show for every program c and states $\sigma_{1} ,\; \sigma_{2}$ with\\
	\indent\indent $\bullet$ \quad $\sigma_{1} =_\dep(c) \sigma_{2}$\\
	\indent\indent $\bullet$ \quad $\left\langle c,\; \sigma_{1} \right\rangle \rightarrow \sigma^{'}_{1}$ and\\
	\indent\indent $\bullet$ \quad $\left\langle c, \; \sigma_{2}\right\rangle \rightarrow \sigma^{'}_{2}$\\
	\indent that $\sigma^{'}_{1} =_{\Mod(c)}\sigma^{'}_{2}$.\\
	\indent This will be shown by induction on the height h of derivation trees.\\\\
	\subsubsection*{\underline{Induction Base: (h=1)}}
	
	\indent\indent If the derivation tree has height 1 only two derivations are possible, namely \indent the skip and the assignment derivations.\\\\
	\indent \textbf{case:} $c=\text{\textbf{skip}}$\\
	\indent\indent This case is trivial due to the definition of \textbf{mod} and that the empty set is \indent\indent identical in any two arbitrary but fixed states $\sigma_{1}$ and $\sigma_{2}$.\\\\
	\indent \textbf{case:} $c=$ {\boldmath$x:=a$}\\
	\indent\indent Following the definitions of \textbf{dep} and \textbf{mod} we get $\Mod(c)=\left\lbrace x\right\rbrace$ and \\\indent\indent $\dep(c)=\FV(a)$.\\\\
	\indent\indent Furthermore we have
	\begin{center}
		\eRelation
			{\eRule{a}{$\sigma_{i}$}{$z_{i}$}}
			{\eRule{$x:=a$}{$\sigma_{i}$}{$\sigma_{i}\left[ x\mapsto z_{i}\right]$}}, $i\in \left\lbrace 1,\; 2\right\rbrace$\\
	\end{center}
	\indent\indent\indent Since $\sigma_{1}=_{\FV(a)}\sigma_{2}$ (assumption) it holds that $\left\langle a,\; \sigma_{1}\right\rangle\rightarrow z \Leftrightarrow \left\langle a,\; \sigma_{2}\right\rangle\rightarrow z$ \indent\indent(Lemma 2.6, Chapter 2, Slide 17). Thus $z_{1}=z_{2}$ and moreover $\sigma^{'}_{1}=_{\Mod(c)}\sigma^{'}_{2}$.
	
	\subsubsection*{\underline{Induction Hypothesis:}}
	
	\begin{center}
		$\sigma_{1} =_{\dep(c)} \sigma_{2}$, $\left\langle c,\; \sigma_{1} \right\rangle \rightarrow \sigma^{'}_{1}$ and $\left\langle c, \; \sigma_{2}\right\rangle \rightarrow \sigma^{'}_{2}$ imply that $\sigma^{'}_{1} =_{\Mod(c)}\sigma^{'}_{2}$
	\end{center}
	\indent\indent holds for all derivations of an arbitrary, but fixed height h and for all states \indent$\sigma,\; \sigma^{'}$.
	
\newpage	
	
	\subsubsection*{\underline{Induction Step: ({\boldmath$h\mapsto h+1$})}}
	
	\indent\indent \textbf{case:} $c=${\boldmath $c_{1};c_{2}$}\\
	
	\indent\indent Following the definition of \textbf{dep} and \textbf{mod} we get\\\indent\indent $\Mod(c)=\Mod(c_{1})\cup \Mod(c_2)$ and $\dep(c)=\dep(c_{1})\cup \dep(c_{2})$.\\
	
	\indent\indent Furthermore we have
	\begin{center}
		\eRelation
			{\eRule{$c_{1}$}{$\sigma_{i}$}{$\sigma_{i}^{*}$} \qquad \eRule{$c_{2}$}{$\sigma_{i}^{*}$}{$\sigma_{i}^{'}$}}
			{\eRule{$c_{1};c_{2}$}{$\sigma_{i}$}{$\sigma_{i}^{'}$}}, $i\in \left\lbrace 1,\; 2\right\rbrace$
	\end{center}
	
	\indent\indent By induction hypothesis it holds that $\sigma_{1}^{*}=_{\Mod(c_{1})}\sigma_{2}^{*}$.\\
	\indent\indent Now let $R=\dep(c_{2})\verb|\|\Mod(c_{1})$. Then we get two additional coherences:\\
	\indent\indent\indent $(1)$ \quad $\sigma_{1}=_{R}\sigma_{2}$ (because $R\subseteq \dep(c)$)\\
	\indent\indent\indent $(2)$ \quad $\sigma_{i}=_{R}\sigma_{i}^{*}$, $i\in \left\lbrace 1,\; 2\right\rbrace$ (by auxiliary (a))\\
	\indent\indent Thus it holds that $\sigma_{1}^{*}=_{R}\sigma_{2}^{*}$ and therefore $\sigma_{1}^{*}=_{\Mod(c_1)\cup\dep(c_2)}\sigma_{2}^{*}$\\
	\indent\indent Applying the induction hypothesis we then get that $\sigma_{1}^{'}=_{\Mod(c_2)}\sigma_{2}^{'}$\\
	\indent\indent Now we introduce another set $R^{'}=\Mod(c_{1})\verb|\|\dep(c_{2})\subseteq\Mod(c_{1})$.\\
	\indent\indent As stated earlier $\sigma_{1}^{*}=_{\Mod(c_{1})}\sigma_{2}^{*}$ thus it also holds that $\sigma_{1}^{*}=_{R^{'}}\sigma_{2}^{*}$.\\
	\indent\indent Using auxiliary (a) we learn that $\sigma_{1}^{'}=_{R^{'}}\sigma_{2}^{'}$ holds.\\
	\indent\indent Using this information and our previously gathered knowledge we now know \indent\indent that $\sigma_{1}^{'}=_{\Mod(c_{1})\verb|\|\dep(c_{2})\cup\Mod(c_{2})}\sigma_{2}^{'}$ holds.\\
	\indent\indent This is equal to $\sigma_{1}^{'}=_{\Mod(c_{1})\cup\Mod(c_{2})}\sigma_{2}^{'}$.\\
	
	\indent \textbf{case:} $c=$ {\boldmath $\text{\textbf{repeat}}\; c ^{'}\; \text{\textbf{until}}\; b$}\\
	
	\indent\indent Following the definition of $\dep$ and $\Mod$ we get $\Mod(c)=\Mod(c^{'})$ and\\\indent\indent $\dep(c)=\dep(c^{'})\cup\FV(b)$.\\\\
	\indent\indent Lets assume there exist states $\sigma_{1}^{*},\;\sigma_{2}^{*}$ so that \eRule{$c^{'}$}{$\sigma_{i}$}{$\sigma_{i}^{*}$} \quad $\left(i\in\left\lbrace 1,\;2\right\rbrace\right)$\\
	\indent\indent By induction hypothesis and auxiliary (a) we know that\\\indent\indent $\sigma_{1}^{*}=_{\dep(c)\cup\Mod(c^{'})}\sigma_{2}^{*}$ holds.\\
	\indent\indent Lemma 2.6 (for boolean expressions) yields that\\\indent\indent \eRule{$b$}{$\sigma_{1}^{*}$}{$\text{\textbf{true}}$} iff \eRule{$b$}{$\sigma_{2}^{*}$}{$\text{\textbf{true}}$}\\
	
	\indent\indent We now assume that \eRule{$b$}{$\sigma_{i}^{*}$}{$\text{\textbf{true}}$}.\\
	\indent\indent This leads to the following derivation tree:
	\begin{center}
		\begin{prooftree}
			\hypo{\left\langle c^{'}, \sigma_{i}\right\rangle \rightarrow \sigma_{i}^{*}}
			\hypo{\left\langle b, \sigma_{i}^{*}\right\rangle \rightarrow \text{\textbf{true}}}
			\infer2[]{\left\langle \text{\textbf{repeat}}\; c^{'}\; \text{\textbf{until}}\; b, \sigma_{i} \right\rangle \rightarrow \sigma_{i}^{*}}
		\end{prooftree}
	\end{center}
	
	\indent\indent Since $\sigma^{*}_{1}=_{\text{\textbf{dep}}(c)\cup\text{\textbf{mod}}(c^{'})}\sigma^{*}_{2}$ (By induction hypothesis and auxiliary (a)),\\\indent\indent $\sigma^{*}_{1}=_{\text{\textbf{mod}}(c)}\sigma^{*}_{2}$ holds as well ($\textbf{mod}(c)=\textbf{mod}(c^{'})$).\\\\
	\indent\indent Now assume that \eRule{$b$}{$\sigma^{*}_{i}$}{$\text{\textbf{false}}$}.\\
	\indent\indent This leads to the following derivation tree:
	\begin{center}
		\begin{prooftree}
			\hypo{\left\langle c,\; \sigma_{i}\right\rangle\rightarrow\sigma^{*}_{i}}
			\hypo{\left\langle b,\; \sigma^{*}_{i}\right\rangle\rightarrow\text{\textbf{false}}}
			\hypo{\cdots}
			\infer1{\left\langle \text{\textbf{repeat}}\; c^{'}\; \text{\textbf{until}}\; b,\; \sigma^{*}_{i}\right\rangle\rightarrow\sigma^{'}_{i}}
			\infer3{\left\langle \text{\textbf{repeat}}\; c^{'}\; \text{\textbf{until}}\; b,\; \sigma_{i}\right\rangle\rightarrow\sigma^{'}_{i}}
		\end{prooftree}
	\end{center}
	
	\indent\indent In order to prove that $\sigma^{'}_{1}=_{\text{\textbf{mod}}(c)}\sigma^{'}_{2}$ we show that $\sigma^{*}_{1}=_{\dep(c)}\sigma^{*}_{2}$.\\\indent\indent The induction hypothesis can than be utilised to prove that $\sigma^{'}_{1}=_{\Mod(c)}\sigma^{'}_{2}$.\\
	\indent\indent By induction hypothesis and auxiliary (a) we get that $\sigma^{*}_{1}=_{\dep(c)\verb|\|\Mod(c^{'})}\sigma^{*}_{2}$. \indent\indent Since the induction hypothesis also states that $\sigma^{*}_{1}=_{\Mod(c^{'})}\sigma^{*}_{2}$ we also get that \indent\indent $\sigma^{*}_{1}=_{\dep(c)}\sigma^{*}_{2}$. Therefore we get that $\sigma^{'}_{1}=_{\Mod(c)}\sigma^{'}_{2}$ by induction hypothesis.
	\begin{flushright}
	$\qed$
	\end{flushright}
	
\end{document}